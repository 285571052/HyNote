% 多元高斯分布的kl距离
\documentclass[UTF8,a4paper]{ctexart}
\pagestyle{empty}

\ctexset{section/format=\Large\bfseries} % 标题左对齐
\usepackage{amsmath} % 使用align
\usepackage[margin=1in]{geometry}

\begin{document}
  \section*{多元高斯分布的kl距离}

  \subsection{问题描述}
    假设$p$和$q$分别表示两个正态分布的概率密度,
    其中$\mu_1$和$\mu2$分别表示$p$和$q$的均值向量,
    $\Sigma_1$和$\Sigma_2$分别表示$p$和$q$的协方差矩阵,
    求$p$和$q$这两个分布之间的\emph{Kullback-Leibler距离}
    是多少?写出推导过程。

  \subsection{结果}
      \[ D_{KL}\left( p\|q \right ) =
        \frac{1}{2} \left( \ln \frac{\left| \Sigma_1 \right|}{\left| \Sigma_2 \right|} - n
        +
        tr
          \left(
              \Sigma_2^{-1}\Sigma_1
          \right)
        +
          \left( \mu_1 - \mu_2 \right)^T\Sigma_2^{-1}\left( \mu_1 - \mu_2 \right)
        \right)
      \]

  \subsection{推导过程}
\begin{small}
  \begin{align}
    D_{KL}\left( p\|q \right ) &= \int_x p(x)\ln \frac{ p(x) }{ q(x) } dx \label{eq1}\\
    &= \int_x p(x)
        \ln
          \left( (2\pi)^{\frac{n - n}{2}}   \right )
            \left( \frac{\left| \Sigma_1 \right|}{\left| \Sigma_2 \right|} \right )^\frac{1}{2}
            \\&\quad
            exp\left(
                \frac{1}{2}\left(
                  \left( x - \mu_2 \right)^T\Sigma_2^{-1}\left( x - \mu_2 \right) -
                  \left( x - \mu_1 \right)^T\Sigma_1^{-1}\left( x - \mu_1 \right)
                \right )
          \right )
        dx\label{eq2}\\
    &=  \frac{1}{2}
          \int_x p(x)
            \left(
              \ln \frac{\left| \Sigma_1 \right|}{\left| \Sigma_2 \right|} +
              \left(
                    \left( x - \mu_2 \right)^T\Sigma_2^{-1}\left( x - \mu_2 \right) -
                    \left( x - \mu_1 \right)^T\Sigma_1^{-1}\left( x - \mu_1 \right)
              \right )
            \right )
          dx\label{eq3}\\
    &=  \frac{1}{2} \ln \frac{\left| \Sigma_1 \right|}{\left| \Sigma_2 \right|} +
          \frac{1}{2}\int_x p(x)
                \left(
                \left( x - \mu_2 \right)^T\Sigma_2^{-1}\left( x - \mu_2 \right) -
                \left( x - \mu_1 \right)^T\Sigma_1^{-1}\left( x - \mu_1 \right)
                \right )
          dx\label{eq4}\\
    &=  \frac{1}{2} \ln \frac{\left| \Sigma_1 \right|}{\left| \Sigma_2 \right|} +
        \frac{1}{2} E_p
          \left[
            \left( x - \mu_2 \right)^T\Sigma_2^{-1}\left( x - \mu_2 \right) -
            \left( x - \mu_1 \right)^T\Sigma_1^{-1}\left( x - \mu_1 \right)
          \right]\label{eq5} \\
    &=  \frac{1}{2} \ln \frac{\left| \Sigma_1 \right|}{\left| \Sigma_2 \right|}
        -
        \frac{1}{2} E_p
          \left[
              tr\left( \Sigma_1^{-1}\left( x - \mu_1 \right)\left( x - \mu_1 \right)^T \right)
          \right]
        +
        \frac{1}{2} E_p
          \left[
              tr\left( \Sigma_2^{-1}\left( x - \mu_2 \right)\left( x - \mu_2 \right)^T \right)
          \right]\label{eq6}\\
    &= \frac{1}{2} \ln \frac{\left| \Sigma_1 \right|}{\left| \Sigma_2 \right|}
       -
       \frac{1}{2} tr
         \left(
             E_p\left[ \Sigma_1^{-1}\left( x - \mu_1 \right)\left( x - \mu_1 \right)^T \right]
         \right)
      +
      \frac{1}{2} tr
        \left(
            E_p\left[ \Sigma_2^{-1}\left( x - \mu_2 \right)\left( x - \mu_2 \right)^T \right]
        \right)\label{eq7}\\
    &= \frac{1}{2} \ln \frac{\left| \Sigma_1 \right|}{\left| \Sigma_2 \right|}
      -
      \frac{1}{2} tr
        \left(
            \Sigma_1^{-1}E_p\left[ \left( x - \mu_1 \right)\left( x - \mu_1 \right)^T \right]
        \right)
     +
     \frac{1}{2} tr
       \left(
           \Sigma_2^{-1}E_p\left[ \left( x - \mu_2 \right)\left( x - \mu_2 \right)^T \right]
       \right)\label{eq8}\\
   &= \frac{1}{2} \ln \frac{\left| \Sigma_1 \right|}{\left| \Sigma_2 \right|}
     -
     \frac{1}{2} tr
       \left(
           \Sigma_1^{-1}\Sigma_1
       \right)
    +
    \frac{1}{2} tr
      \left(
          \Sigma_2^{-1}E_p\left[ xx^T - x\mu_2^T - \mu_2x^T + \mu_2\mu_2^T \right]
      \right)\label{eq9}\\
    &= \frac{1}{2} \ln \frac{\left| \Sigma_1 \right|}{\left| \Sigma_2 \right|} - \frac{n}{2}
     +
     \frac{1}{2} tr
       \left(
           \Sigma_2^{-1}\left( \Sigma_1 + \mu_1\mu_1^T - \mu_1\mu_2^T - \mu_2\mu_1^T + \mu_2\mu_2^T \right)
       \right)\label{eq10}\\
    &= \frac{1}{2} \ln \frac{\left| \Sigma_1 \right|}{\left| \Sigma_2 \right|} - \frac{n}{2}
      +
      \frac{1}{2} tr
        \left(
            \Sigma_2^{-1}\Sigma_1 + \Sigma_2^{-1}\left( \mu_1\mu_1^T - \mu_1\mu_2^T - \mu_2\mu_1^T + \mu_2\mu_2^T \right)
        \right)\label{eq11}\\
    &= \frac{1}{2} \ln \frac{\left| \Sigma_1 \right|}{\left| \Sigma_2 \right|} - \frac{n}{2}
      +
      \frac{1}{2} tr
        \left(
            \Sigma_2^{-1}\Sigma_1 + \Sigma_2^{-1}\left( \mu_1 - \mu_2 \right)\left( \mu_1 - \mu_2 \right)^T
        \right)\label{eq12}\\
    &= \frac{1}{2} \ln \frac{\left| \Sigma_1 \right|}{\left| \Sigma_2 \right|} - \frac{n}{2}
      +
      \frac{1}{2} tr
        \left(
            \Sigma_2^{-1}\Sigma_1
        \right)
      +
      \frac{1}{2} tr
        \left(
            \Sigma_2^{-1}\left( \mu_1 - \mu_2 \right)\left( \mu_1 - \mu_2 \right)^T
        \right)\label{eq13}\\
    &= \frac{1}{2} \left( \ln \frac{\left| \Sigma_1 \right|}{\left| \Sigma_2 \right|} - n
      +
      tr
        \left(
            \Sigma_2^{-1}\Sigma_1
        \right)
      +
      tr
        \left(
            \Sigma_2^{-1}\left( \mu_1 - \mu_2 \right)\left( \mu_1 - \mu_2 \right)^T
        \right)
      \right)\label{eq14}\\
    &= \frac{1}{2} \left( \ln \frac{\left| \Sigma_1 \right|}{\left| \Sigma_2 \right|} - n
      +
      tr
        \left(
            \Sigma_2^{-1}\Sigma_1
        \right)
      +
        \left( \mu_1 - \mu_2 \right)^T\Sigma_2^{-1}\left( \mu_1 - \mu_2 \right)
      \right)\label{eq15}
  \end{align}
\end{small}

\begin{itemize}
  \item[\eqref{eq1}] \emph{Kullback-Leibler距离}的定义
  \item[\eqref{eq5}] 满足期望的形式\\
      \begin{quote}\[
        \boxed{E\left( g(x) \right) = \int_x p(x)g(x)\,dx}
      \]\end{quote}
  \item[\eqref{eq6}] $
    \left( x - \mu_2 \right)^T\Sigma_2^{-1}\left( x - \mu_2 \right)  =
    tr\left( \Sigma_1^{-1}\left( x - \mu_1 \right)\left( x - \mu_1 \right)^T \right)$\\
    为方便,简记为$A^TBA = tr(BAA^T)$\\
    \begin{small}\begin{align}
        A^TBA
        &=
          \begin{bmatrix}
            a_1 &  a_2 &  \dots & a_n
          \end{bmatrix}
          \begin{bmatrix}
            b_{11}  & \cdots & b_{1n} \\
            \vdots  & b_{ij} & \vdots \\
            b_{n1}  & \cdots & b_{nn}
          \end{bmatrix}
          \begin{bmatrix}
            a_1 \\  a_2 \\  \vdots \\ a_n
          \end{bmatrix}\\
        &=
          \begin{bmatrix}
            \sum_{i = 1}^n a_ib_{i1} & \cdots & \sum_{i = j}^n a_ib_{ij} & \cdots \sum_{i = j}^n a_ib_{in}
          \end{bmatrix}
          \begin{bmatrix}
            a_1 \\  a_2 \\  \vdots \\ a_n
          \end{bmatrix}\\
        &= \sum_{j = 1}^n a_j \sum_{i = 1}^n a_ib_{ij} \\
        &= \sum_{j = 1}^n \sum_{i = 1}^n a_ja_ib_{ij} \\
        &= \sum_{i = 1}^n \sum_{j = 1}^n \left (a_ia_j \right )b_{ji} \\
        &= \sum_{i = 1}^n \sum_{j = 1}^n \left (AA^T\right)_{ij}B_{ji} \\
        &= tr\left (B\left(AA^T\right)\right)\\
        &= tr\left (BAA^T\right) \label{eq16}
    \end{align}\end{small}
  \item[\eqref{eq7}] $\boxed{tr\left( E\left[ x\right] \right) =  E\left[tr\left( x \right) \right]}$\\
  \item[\eqref{eq8}] $\boxed{E_p\left[ \left( x - \mu_1 \right)\left( x - \mu_1 \right)^T \right] = \Sigma_1}$\\
    协方差的定义,注意后面的$E_p\left[ \left( x - \mu_2 \right)\left( x - \mu_2 \right)^T \right] \neq \Sigma_2$
  \item[\eqref{eq9}] $\boxed{E(xx^T) = \Sigma + \mu\mu_T}$\\
  \item[\eqref{eq15}]  同\eqref{eq6}
  \item[\eqref{eq16}]
    \[\boxed{ tr\left( AB\right) = \sum_{i = 1}^n(AB)_{ii} = \sum_{i = 1}^n\sum_{j = 1}^nA_{ij}B_{ji} }\]
\end{itemize}
\end{document}
