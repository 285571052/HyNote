\documentclass[UTF8,a4paper]{ctexart}
\usepackage[margin=1in]{geometry}
\usepackage{fancyhdr,hyperref,amsmath}
\pagestyle{fancy}
\hypersetup{hidelinks}

\lhead{\bfseries \leftmark}
\chead{}
\rhead{SCUT}
\lfoot{\url{https://github.com/285571052}}
\cfoot{qhy}
\rfoot{\thepage}
\setlength{\headheight}{13pt}
\renewcommand{\headrulewidth}{0.4pt}
\renewcommand{\footrulewidth}{0.4pt}

\setlength{\parindent}{0pt}
\newcommand{\spaceline}{\vspace{\baselineskip}}

\author{ qhy }
\date{\today}
\title{软件工程}

\begin{document}
  \maketitle
  \tableofcontents
  \newpage

  \section{介绍}
  "软件"的概念、原则、方法、思想

  摩尔定律?软件危机?

  \textbf{什么是软件工程?}

  工程:起源于建筑行业,与建筑行业类似,从低到高遵循一定方法进行搭建,从工程的角度(使用工程的思想)对软件开发进行管理。

  \textbf{软件工程包括两大部分}
  \begin{itemize}
    \item [1.] 方法论

      软件生命周期
    \item [2.] 建模

      包括需求分析和软件设计
  \end{itemize}

  \textbf{为什么需要软件工程?}

  因为软件危机(硬件快速发展,软件发展滞后,且开发出的软件容易出问题,开发成本高,可维护性差),
  所以需要软件工程。主要体现为规范化和文档化。

  注:规范化和文档化,在各个阶段的体现并不完全相同。

  瀑布模型,60-70年代,重量级开发方法

  之后逐渐降低对文档的要求(通过规范来降低文档的要求)

  \textbf{软件工程的关键是使用 UML 进行建模}

  UML于97年提出,是公认的标准规范描述

  \section{软件危机}
  \textbf{软件危机:}
  \begin{itemize}
    \item 成本高
    \item 软件质量得不到拨正
    \item 计算机软硬费用比
    \item 进度难以控制
    \item 维护困难
  \end{itemize}

  软件危机无非是关于时间、成本、质量这三个方面的问题。

  \spaceline
  质量包括可靠性、安全性、可维护性、可移植性等方面。

  \spaceline
  \textbf{可靠性:}99.99\%,表示一年之内正常工作的概率是99.99\%,或一年之内故障的概率是0.01\%

  \spaceline
  \textbf{软件危机的解决办法:}主要分为两个方面:

  $\left \{\begin{array}{l}
    \text{从管理的角度}\left \{\begin{array}{l}
    \text{文档的标准化}\\
    \text{软件开发的过程的研究}\\
    \text{人们的交流方式}
    \end{array}\right .
    \\
    \text{软件开发方法的研究} \left \{ \begin{array}{l}
    \text{结构化开发方法}\\
    \text{面向对象开发方法}
    \end{array} \right .
  \end{array}\right .$

  管理是影响软件项目成功开发的全局性因素,而技术只是影响局部。\spaceline

  \textbf{结构化与面向对象的区别?}\\
  解决问题分为两个阶段:
  \begin{itemize}
    \item 问题分析
    \item 求解(编码)
  \end{itemize}

  问题分析中,结构化与面向对象都采用相同的层次化的思想,都是自顶向下逐层细化问题。\\
  而两者的区别主要在于编码部分,结构化采用自底向上的编码方式,而面向对象则是采用从中间向两边编码的方式。

  \spaceline
  \textbf{软件工程的三个要素:}
  \begin{itemize}
    \item 工具
    \item 方法
    \item 过程\\
    工具与方法的综合使用
  \end{itemize}

  \spaceline
  $\text{软件生命周期}\left \{\begin{array}{l}
  \text{软件定义} \left \{\begin{array}{l}
    \text{问题分析}\\
    \text{可行性研究}\\
    \text{需求分析}
  \end{array} \right .\\
  \text{软件开发} \left \{\begin{array}{l}
    \text{总体设计,单元测试}\\
    \text{详细设计,综合测试}\\
    \text{编码}
  \end{array} \right .\\
  \text{运行维护:持续满足用户需求}
  \end{array} \right .$

  \spaceline
  \textbf{描述bug的术语:}
  \begin{itemize}
    \item 错误\\
    当人们在进行软件开发活动的过程中,出错时,例如设计人员错误理解用户需求
    \item 故障\\
    非正常或非期望的工作状态(比如异常?)
    \item 失效\\
    指系统违背了它应有的行为
  \end{itemize}

  \spaceline
  \textbf{软件质量的各种视角:}(\color{red} ppt)

  个人利益不同,评价标准也不同。

  \spaceline
  \textbf{什么是好的软件?}包括以下三个方面
  \begin{itemize}
    \item 产品的质量\\
    产本本身的质量
    \item 过程的质量\\
    决定一个公司是否能持续的产出高质产品
    \item 商业环境背景下产品的质量
  \end{itemize}

\section{UML}

\textbf{UML体系结构视图:}
\begin{itemize}
  \item 结构
  \item 行为
  \item 分组
  \item 注释
\end{itemize}

\textbf{UML基本构成:}
\begin{itemize}
  \item 基本模型元素
  \item 关系
  \item 模型图
\end{itemize}

\textbf{行为:}
\begin{itemize}
  \item 交互\\
  多对象的行为
  \item 状态机\\
  单个对象自身状态的变化
\end{itemize}

\textbf{关系:}
\begin{itemize}
  \item 关联关系\\
  相互存在
  \item 依赖关系\\
  单向
  \item 泛化\\
  一般表现为继承关系,但泛化实际范围更广,它不要求两者之间有某种继承,只需要是特殊和一般的对比即可
  \item 实现关系\\
  比如:类与接口
\end{itemize}

\textbf{9种模型图:}...\\
也有说分10种,第10种是包图,它实际上是作用于前9种图使得更方便管理

\textbf{UML的建模规则:}...

\textbf{UML公共机制:}
\begin{itemize}
  \item 规则说明
  \item 通用划分
  \item 修饰
  \item 扩展机制
\end{itemize}

\textbf{扩展机制:}
\begin{itemize}
  \item 构造型\\
  构造新的元素
  \item 标记值\\
  构造型元素的属性(这个属性是更一般的属性,不一定是对象自身的特性,比如 版本号)
  \item 约束
\end{itemize}

物理层上信号的传输:信号
有两类:
1. 模拟信号
2. 数字信号

信号在传输的过程中,接收方接受到的信号可能是衰减和变形的(失真)

截至频率fc,一般来说0~fc这一段频率,振幅在传输的过程中不会明显衰减

物理带宽:传输过程中振幅不会明显衰减的频率范围{这个频率是什么?}

数字带宽:单位时间内传输的信息的总量

物理带宽和数字带宽的关系是?

奈奎斯特定理:
在无噪声信道中,当带宽为B Hz(物理带宽),信号电平为V级,则最大传输速率(数字带宽)=2Blog_2 V (bps,一秒钟信号变化的次数,物理带宽和波特率的关系?)
其中,V为信号的点评级数,在二进制中卫0、1两级

而物理带宽在信道确定的时候也跟着确定,因此要提高数字带宽,只能提高电平级数。

{这个公式怎么理解?最多2B次采样什么意思,V级别什么意思?2B是采样率/波特率,B Hz的最大采样率为2B Hz}

香浓定理

两道习题?

/****************/
传输介质
1. 铜线
	1. 同轴电缆
	2. 双绞线
		1. UDP.非屏蔽双绞线:成本低,尺寸小,易于安装,但易受敢逃,传输距离性能,收到绞距,10~100Mbps、最大传输距离100m(短)
		2. STP,屏蔽双绞线(4对线,每对线都有屏蔽层,最外面还有一层屏蔽层),成本高,安装不容易(最大距离100m,10~1000=Mbps,
		3. 网屏式双绞线,保留最外的屏蔽层
		除了屏蔽性能,其他是一样的

		双绞线的线序:568B,568A(直通线,两根头的线序相同,交叉线,线序相反)
	3. 电力线
2. 光纤
	3层,两层玻璃纤维,最外层为防护层,重量轻,损耗低,不受电磁辐射干扰,传输带宽大但昂贵易断
	原理:全反射,因而损耗低
	单模光纤:单一模式传输(平行入射),运行波长850~1300nm,纤芯细8~10um,激光产生但束光,高带宽,长距离
	多模光纤:多个模式同时传输(多个角度入射,只要大于临界角度,运行波长1310nm或1550n,纤芯粗50~62.5um,LED产生多束光。低带宽,短距离

	光纤连接:光纤连接器损失10~20%,机械拼接,特殊的套管加紧(10%),熔合几乎无损失

3. 无线电,卫星,激光

/***********************/
复用技术:让多个用户共享同一根信道
FMD:频分多路复用:将频谱分成若干段,每个用户占据一段来传输自己的信号

OFDM,正交FDM,相邻两个波之间可以重叠,

TDM:时分多路复用,用户分时轮流使用,

stdm:统计时分多路复用,动态分配带宽,

CDM,码分多路复用,

时分多路复用的例子:。。。
每个用户有一个码片序列,两两正交

WDM,波分多路复用,本质跟FDM一样

TDM,FDM的例子。。。

/******************************/
调制技术
基带传输:直接把数据比特转换成信号
编码方式:
1. 高电平为1,低电平为0
2. 电平不变表示0,跳变表示1
3. 曼彻斯特编码(局域网,以太网),高电压跳变到低电压表示1,低电压到高电压表示0
4. 双极性编码:0始终用一个中间电压表示,1使用高电压表示之后使用低电压表示

通带传输:通过调节振幅、相位,频率来传输比特
1. 调幅,使用不同振幅表示信号,振幅为1表示1,没有振幅表示0
2. 调频,使用不同频率载波信号表示振幅
3. 调相:使用不同相位表示0和1

往往是使用几种方案结合,条幅和调相相结合
码元:承载信息量的基本信号单位,常使用时间间隔相同的符号来表示二进制数字

计算的例子:。。。

/*********************************/
公共交换电话网络:PSTN
PSTN的组成??????

本地回路:用户到交换局
调制解调器,数字信号-》模拟信号

为什么56kbps?

xDSL,modem带宽低,xdsl本地回路使用全部的1.1Mhz,宽带能达到8M

FTTH:光纤到户

干线:复用技术

PCM:模拟信号-》数字信号

交换:
电路交换:双方打通通道

报文交换

分组交换:每个分组独立寻径,直接投放数据

/******************************/
物理层部件/设备
被动设备:
接线板,插头,插座,电缆

主动设备:
转发器(网卡的部分),中继器(再生信号:去噪和放大信号),集线器

可能冲突!

\end{document}
