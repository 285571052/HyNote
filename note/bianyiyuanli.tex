\documentclass[UTF8,a4paper]{ctexart}
\usepackage[margin=1in]{geometry}
\usepackage{fancyhdr,hyperref}
\pagestyle{fancy}
\hypersetup{hidelinks}

\lhead{\bfseries \leftmark}
\chead{}
\rhead{SCUT}
\lfoot{\url{https://github.com/285571052}}
\cfoot{qhy}
\rfoot{\thepage}
\setlength{\headheight}{13pt}
\renewcommand{\headrulewidth}{0.4pt}
\renewcommand{\footrulewidth}{0.4pt}

\author{ qhy }
\date{\today}
\title{编译原理}

\begin{document}
  \maketitle
  \tableofcontents
  \newpage

  \section{介绍}

  \textbf{预期收货}:
  \begin{itemize}
    \item 通过学习编译原理,写出更高效的代码
    \item 针对目标,自编写编译器

    对某个模型机的编译器进行设计
  \end{itemize}

  \textbf{编译器和解释器的区别?}
  编译器是转换器,解释器是执行系统。

  程序编译的过程主要分为两大阶段:
  \begin{itemize}
    \item [1.] 查错
    \begin{itemize}
      \item 词法分析
      \item 语法分析
      \item 语义分析
    \end{itemize}

    \item [2.] 综合(翻译)
    \begin{itemize}
      \item [1.] 产生中间代码进一步优化
      \item [2.] 目标代码生成
    \end{itemize}
  \end{itemize}

  高级程序处理过程:初始源程序$\to$预处理$\to$源程序$\to$编译$\to$目标汇编$\to$机器代码

  注:生成机器代码的时候,并不是直接生成,而是先生成对应的汇编代码,再生成机器代码。

  编译过程:每个阶段的输出作为下一个阶段的输入(即数据从一种形式转换成另一种形式)。

  编译过程的每个阶段都包括两个相同的处理:表格管理和出错处理。

  表格管理是保存编译过程每个阶段的数据和结果,出错处理则是对编译过程遇到的语法,词法等错误进行处理。

    \subsection{词法分析}
    主要分为两大步骤,扫描和分解

    扫描为从左到右扫描

    分解为以介符对语句进行分割(注:双引号内的介符不可划分)

    最后结果使用一个二元组保存,格式为(种类,值)

    \subsection{语义分析}
    语义分析:不断构造语法树

    如:类型不对,数组越界等

    过程:单词符号串$\to$语法分析$\to$语法短语

    识别规则:描述程序结果的规则,通常由递归规则表示。

    输出:合法的语法树

    \subsection{语义分析}
    语义分析:包括静态语义和动态语义

    常见的错误包括类型不匹配,数组越界等

    输出:生成中间代码

    结果使用四元式表示,格式为:(运算符,运算对象1,运算对象2,结果)

    \subsection{代码优化}
    代码优化:对代码进行优化化简

    \subsection{目标代码生成}
    目标代码生成:与目标机器紧紧相关,先生成对应的汇编代码,再生成机器代码

    \subsection{表格管理与出错处理}
    表格管理与出错处理:每个阶段都执行这一操作

    \subsection{编译程序划分}
    编译程序划分:分析与综合(翻译)两个阶段

    按是否与目标机器相关:前端(无关)和后端(相关)

\end{document}
