\documentclass[UTF8,a4paper]{ctexart}
\usepackage[margin=1in]{geometry}
\usepackage{fancyhdr,hyperref}
\pagestyle{fancy}
\hypersetup{hidelinks}

\lhead{\bfseries \leftmark}
\chead{}
\rhead{SCUT}
\lfoot{\url{https://github.com/285571052}}
\cfoot{qhy}
\rfoot{\thepage}
\setlength{\headheight}{13pt}
\renewcommand{\headrulewidth}{0.4pt}
\renewcommand{\footrulewidth}{0.4pt}

\author{ qhy }
\date{\today}
\title{软件工程}

\begin{document}
  \maketitle
  \tableofcontents
  \newpage

  \section{介绍}
  "软件"的概念、原则、方法、思想

  摩尔定律?软件危机?

  \textbf{什么是软件工程?}

  工程:起源于建筑行业,与建筑行业类似,从低到高遵循一定方法进行搭建,从工程的角度(使用工程的思想)对软件开发进行管理。

  \textbf{软件工程包括两大部分}
  \begin{itemize}
    \item [1.] 方法论

      软件生命周期
    \item [2.] 建模

      包括需求分析和软件设计
  \end{itemize}

  \textbf{为什么需要软件工程?}

  因为软件危机(硬件快速发展,软件发展滞后,且开发出的软件容易出问题,开发成本高,可维护性差),
  所以需要软件工程。主要体现为规范化和文档化。

  注:规范化和文档化,在各个阶段的体现并不完全相同。

  瀑布模型,60-70年代,重量级开发方法

  之后逐渐降低对文档的要求(通过规范来降低文档的要求)

  \textbf{软件工程的关键是使用 UML 进行建模}

  UML于97年提出,是公认的标准规范描述







\end{document}
